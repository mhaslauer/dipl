\section*{Kurzzusammenfassung}
Die beiden Begriffe DevOps und Continuous Delivery sind im Web-Bereich der IT Branche aktuell sehr stark vertreten. Immer mehr Unternehmen führen DevOps in ihre Prozesse ein und betreiben Continuous Delivery für ihre Online-Plattform. Die genaue Definition dieser beiden Begriffe ist allerdings unklar und nicht genau festgelegt. Die vorliegende Arbeit zeigt auf, was diese beiden Begriffe bedeuten und welche Vorteile bzw. Nachteile die Umsetzung dieser Konzepte für ein Web-Unternehmen hat. Zur Veranschaulichung werden diese Konzepte, im Rahmen einer Fallstudie, für ein Unternehmen diskutiert. Grundsätzlich geht es bei DevOps und Continuous Delivery um die vollständige Automatisierung der Build und Deployment Pipeline von Software Projekten. Damit soll bei gleichbleibender Effektivität die Effizienz erheblich gesteigert werden. Es wird aufgezeigt, welche Änderungen sich dadurch für ein Unternehmen ergeben und wie diese durchgeführt werden können. Insbesondere werden dabei die Bereiche der Prozesse, Infrastruktur und Firmenkultur betrachtet. In vorliegender Arbeit wird dazu eine minimale Infrastruktur aufgebaut, die benötigt wird um DevOps und Continuous Delivery betreiben zu können. Anschließend wird anhand vier ausgewählter Applikationen gezeigt, wie eine Umstellung durchgeführt wird  und der dafür notwendige Arbeitsaufwand erfasst. Die positiven Effekte nach der Umstellung werden anhand von definierten Metriken erfasst und mit dem Zustand vor der Einführung verglichen. Im Zuge vorliegender Arbeit wird schlussendlich aufgezeigt, dass die Effizienz erheblich verbessert wird und sich der notwendige Arbeitsaufwand in Grenzen hält. Die Einführung von DevOps und Continuous Delivery ist deshalb für viele Unternehmen erstrebenswert.


\newpage
\section*{Abstract}
DevOps and Continuous Delivery are currently two common terms in the web branch of the IT area. More and more companies adopt the principles of DevOps and try to do Continuous Delivery in the development process of their online platforms. Nevertheless the concrete definitions of these two terms are unclear. This master thesis shows in depth what these two terms stand for and what implementation of these principles implies for a web company. In order to achieve this, a case study is setup to show the implementation by means of a concrete company. Basically the adoption of DevOps and Continuous Delivery result in a complete automation of the build and deployment pipeline of software projects. Therefore at constant effectivity the efficiency should rise dramatically. In this case study is shown which changes have to be done in the area of processes, infrastructure and company culture and how these changes can be performed. Therefore in the course of this master thesis, a minimal infrastructure will be implemented that is necessary to achieve DevOps and Continuous Delivery. Afterwards the automation of the build and deployment pipeline is shown on behalf of four selected applications and how much effort is involved. The positive effects after the implementation are measured by defined metrics and compared to the state before the implementation started. This shows how the efficiency increases for these applications and how many effort is neccessary to achieve it. As a conclusion the adoption of the principles DevOps and Continuous Delivery is desirable for many companies.