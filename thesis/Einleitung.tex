\chapter{Einleitung}
In der Softwarebranche ist es sinnvoll standardisierte Prozesse zu schaffen, um die Entwicklung von Software, sowie deren Einsatz möglichst effizient zu gestalten. Dabei müssen viele unterschiedliche Bereiche beachtet werden. Dies umfasst beispielsweise die Entwicklung von Prozessen für die Planung und Umsetzung von Software, sowie Vorgehensmodelle für den zuverlässigen Einsatz der fertigen Applikationen. Auch zwischenmenschliche Aspekte, wie die Zusammenstellung beteiligter Personen zu einem hochperformanten Team, müssen mit einbezogen werden. In der klassischen Software-Entwicklung gibt es bereits viele Modelle die versuchen, diese Aufgaben zu bewältigen. Die im letzten Jahrzehnt eingesetzten Modelle haben aber mehrere Nachteile \cite{humble2010}. So ist es oft üblich, dass Release-Zyklen mehrere Monate dauern, um den erhöhten Verwaltungs- und Kommunikationsaufwand dabei zu vermeiden. Die durch den langen Zeitraum resultierenden vielfältigen Änderungen, können an vielen Stellen Fehler erzeugen. Auch das Testen der Änderungen erfordert einen hohen Aufwand und benötigt entsprechende Zeit. Auf Grund dieser Faktoren entsteht bei einem Release oft erhöhter Stress für die beteiligten Personen, da es unklar ist, ob die Software noch korrekt betrieben werden kann oder ob es unentdeckte Fehler gibt. Eine gängige Praxis ist, dass die Auslieferung und der Betrieb einer fertigen Applikation oft noch mit viel manueller Arbeit verbunden ist. Diese Vorgehensweise ist zeitaufwändig und fehleranfällig. Darüber hinaus kann nicht ausreichend flexibel auf geänderte Anforderungen (z.B. vom Kunden) reagiert werden. Es ergeben sich lange Durchlaufzeiten, bis Änderungen an der Software tatsächlich eingesetzt werden können. Dies beeinflusst die Zufriedenheit des Kunden oft negativ. 

Ein weiterer Nachteil ist, dass Entwickler sehr spät Rückmeldung zur Qualität ihrer Arbeit bekommen. Im Bereich der Planung und Durchführung von Software-Entwicklung setzen deswegen immer mehr Unternehmen auf sogenannte \textit{Agile Vorgehensmodelle}. Diese agilen Modelle sprechen viele Probleme von klassischen Vorgehensmodellen an. Ihr Hauptziel ist es, Durchlaufzeiten so kurz wie möglich zu halten und dadurch möglichst hohe Flexibilität zu erreichen. Durch kurze Release-Zyklen können Verbesserungen rascher an die Kunden geliefert werden. Agile Vorgehensmodelle fokussieren sich hauptsächlich auf die Entwicklung von Software und bieten Methoden, um diese effizienter zu gestalten. Der sogenannte \textit{Operations-Bereich}, der sich mit der Auslieferung (Deployment) sowie dem zuverlässigen Betrieb der Software beschäftigt, wird hierbei meist nicht beachtet. Deswegen ist es für sie oft eine Herausforderung, mit den häufigen Releases der Entwicklung mithalten zu können. Dies führt letztendlich dazu, dass sie nur schwer einen einwandfreien Betrieb gewährleisten können \cite{humble2010, humble2014}.

\section{Grundkonzepte von DevOps}
DevOps, Verschmelzung der Begriffe \textbf{Dev}elopment und \textbf{Op}eration\textbf{s}, adressiert oben genannte Probleme und bietet Konzepte für deren Lösung. Laut \cite{wolff2014} sollte man zwei Hauptbereiche betrachten, bei denen Änderungen notwendig sind. Konkret handelt es sich dabei um die technische Unterstützung und die Unternehmenskultur:

Der erste Teil beschäftigt sich mit der technischen Unterstützung von Prozessen und Aufgaben. Dadurch soll die Auslieferung von Software effizienter und weniger fehleranfällig gemacht werden. Darüber hinaus wird im Betrieb eine Steigerung der Zuverlässigkeit angestrebt. Wie bereits erwähnt, wird das Deployment einer neuen Software oft manuell durchgeführt. Diese manuelle Tätigkeit muss bei jedem Release vom Operations-Team durchgeführt werden. Dabei wird die neue Version in allen vorhandenen Umgebungen (meist Test, Staging und Produktion) deployed und anschließend getestet, ob die Software noch einwandfrei funktioniert. Bei verkürzten Release-Zyklen muss dies häufiger erfolgen, wodurch die Gefahr besteht, dass das Operations-Team diese Aufgabe nicht mehr sorgfältig und gewissenhaft durchführen kann. Dies hat eine höhere Fehleranfälligkeit zur Folge. Der technische Aspekt von DevOps befasst sich damit, manuelle Tätigkeiten zu minimieren und möglichst viele Arbeitsschritte zu automatisieren. Die Automatisierung umfasst dabei nahezu alle Bereiche einer Software-Entwicklung. So können zum Beispiel die Erstellung des fertigen Release-Paketes, das Testen der Software (speziell bei Regressionstests), sowie das Deployment automatisiert werden. Da die automatisierten Prozesse jedes mal genau gleich durchgeführt werden, wird die Möglichkeit eines menschlichen Fehlers minimiert. Des Weiteren können wesentlich kürzere Durchlaufzeiten beim Deployment erzielt werden. Dieser positive Effekt tritt vor allem bei Clustern in Erscheinung, wo eine Software nicht nur auf einer einzigen Maschine, sondern auf mehreren Servern eingesetzt wird. Durch Automatisierung können alle Server gleichzeitig aktualisiert werden.

Der zweite Bereich von DevOps befasst sich mit der Zusammenarbeit der unterschiedlichen Teams, die im Lebenszyklus einer Software beteiligt sind. Wie bereits erwähnt, bedeutet der Begriff DevOps die Zusammenführung von Entwicklung und Betrieb. Diese beiden Teams verfolgen oft unterschiedliche Ziele und bestehen aus Personen mit verschiedenen Backgrounds und Wissen. Grundsätzlich kennt niemand die Software besser, als derjenige, der sie entwickelt hat. Das Operations-Team ist an der Entwicklung nicht beteiligt, muss aber für den fehlerfreien Betrieb der Software sorgen. Dies erfordert einen hohen Aufwand an Wissenstransfer, von der Entwicklung zum Betrieb. Um dieses Problem zu lösen, versucht DevOps die strikte Trennung beider Abteilungen aufzuheben. Die Mitarbeiter beider Teams sollen in allen Phasen beteiligt sein. Operations stellt die nötigen Rahmenbedingungen her, um Software fehlerfrei betreiben zu können. Die Entwicklung selbst ist am Prozess des Software-Betriebs beteiligt und hat auch die nötigen Befähigungen und Berechtigungen, um ihre Software deployen und betreiben zu können. DevOps verfolgt das Ziel, einen hohen Grad an Kollaboration zwischen den einzelnen Teams zu erreichen \cite{walls2013}.

\section{Continuous Delivery}
Continuous Delivery ist eine Praxis, die mit Hilfe von DevOps ermöglicht wird. Ist in allen erforderlichen Bereichen ein ausreichend hoher Grad an Automatisierung erreicht, kann jede Änderung an der Software sofort durch den Deployment Prozess geschickt werden. Im Laufe dieses Prozesses wird die geänderte Software gebaut, getestet und deployed, was zu einer minimalen Durchlaufzeit führt. Voraussetzung ist, dass die automatisierten Tests eine ausreichend hohe Qualität der Software sicherstellen, um diese im produktiven System einsetzen zu können. Außerdem muss die Automatisierung entsprechend ausgereift sein, damit vertrauenswürdige Ergebnisse erzielt werden können.

Continuous Delivery ist allerdings nicht in allen Bereichen, in denen Software verwendet wird, einsetzbar. Medizinische Software hat beispielsweise sehr hohe Ansprüche an Qualität und Fehlerfreiheit. Hier wird man in vielen Fällen nicht mit rein automatischen Tests auskommen. Dies ist aber ein Ausnahmefall, der auf viele Applikationen, insbesondere bei Internet Plattformen, nicht zutrifft. Hierbei sprechen viele Gründe für den Einsatz von Continuous Delivery.